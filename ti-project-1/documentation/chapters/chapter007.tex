\chapter{Testowanie poprawności i stopnia optymalizacji kodu}

W celu potwierdzenia iż spełnione zostały wymagania niefunkcjonalne, zostały przeprowadzone testy, które mają na celu udowodnić, że strona wraz z jej podstronami spełniają
chociażby standardy \textbf{W3C}. Został też przeprowadzony test \texttt{Lighthouse} dostępny w narzędziach deweloperskich w przeglądarkach opartych na silniku \texttt{Chromium} autorstwa korporacji \textbf{Google}. 

\section{Walidacja HTML oraz CSS - W3C}

Zgodnie z przyjętymi standardami kluczowym elementem weryfikacji i jakości kodu jest poddanie go testowi udostępnionemu przez \textbf{World Wide Web Consortium (W3C)}. Walidacja potwierdzać będzie poprawność semantyczną oraz składniową kodu.
