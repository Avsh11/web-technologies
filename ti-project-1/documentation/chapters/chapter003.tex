\chapter{Opis struktury projektu}

\section{Użyte technologie}
Zgodnie z przyjętymi wymaganiami, strona internetowa \textbf{Virexon} wykonana została tylko i wyłącznie za pomocą języka znaczników \texttt{HTML5} oraz kaskadowych arkuszy stylów \texttt{CSS3} bez żadnych zewnętrznych bibliotek oraz frameworków.

 \texttt{HTML5} wykorzystano do stworzenia podstawowego szkieletu witryny uwzględniając w nim znaczniki takie jak chociażby \texttt{<header>}, \texttt{<nav>}, \texttt{<section>} oraz \texttt{<footer>} co pozytywnie wpływa na pozycjonowanie (SEO) oraz sam układ strony.

 \texttt{CSS} wykorzystano do stworzenia warstwy wizualnej opartej na nowoczesnych standardach, w tym Flexbox do układów oraz zmienne do prostego zarządzania kolorami. Użyte zostały również animacje klatkowe (\texttt{@keyframes}).

\section{Struktura katalogów}
 Struktura katalogów została zaprojektowana z myślą o łatwym zarządzaniu zasobami i ewentualnym rozbudowaniu projektu. Cała struktura przedstawiona została w formie drzewka poniżej:
\vspace{0.5cm} 
\dirtree{%
.1 ti-project-1/.
.2 assets/ \DTcomment{Zasoby graficzne i ikony}.
.3 fotografie.webp \DTcomment{Używane do grafik w treści}.
.3 fotografie.avif \DTcomment{Używane do tła sekcji}.
.3 favicon.svg.
.2 css/ \DTcomment{Arkusze stylów}.
.3 style.css \DTcomment{Główne style strony głównej}.
.3 subsite.css \DTcomment{Style dla podstron}.
.2 html/ \DTcomment{Pliki źródłowe stron}.
.3 index.html \DTcomment{Strona główna}.
.3 about.html.
.3 services.html.
.3 jobs.html.
}
\vspace{0.5cm}
\section{Konwencje kodowania i organizacja stylów}

Aby zapewnić czytelność i łatwość konserwacji kodu, w arkuszach stylów zastosowano następujące konwencje:
\begin{itemize}
    \item \textbf{Zmienne CSS (:root)}: Kolory oraz fonty zostały zdefiniowane w pseudoklasie \texttt{:root}. Umożliwia to szybką zmianę motywu kolorystycznego w jednym miejscu, bez konieczności edycji całego pliku (np. \texttt{--text-color-accent: \#C1121F}).
    \item \textbf{Skalowalne jednostki REM}: W projekcie zastosowano technikę ustawienia bazowej wielkości czcionki dla elementu \texttt{html} na \texttt{50\%}. Dzięki temu 1rem odpowiada 8px (domyślnie przeglądarka ma 16px, 50\% z 16px to 8px).
    \item \textbf{Reset stylów}: Zastosowano globalny reset (\texttt{margin: 0; padding: 0; box-sizing: border-box;}), aby wyeliminować różnice w domyślnym stylowaniu przez różne przeglądarki.
\end{itemize}

\section{Implementacja efektów wizualnych i optymalizacja}

W projekcie duży nacisk położono na płynność interfejsu i nowoczesny wygląd.
\begin{itemize}
    \item \textbf{Optymalizacja renderowania}: W sekcji Hero oraz przy obrazkach wykorzystano właściwość \texttt{transform: translateZ(0)} oraz \texttt{will-change}. Wymusza to na przeglądarce skorzystanie z akceleracji sprzętowej (GPU), co zapobiega "migotaniu" i spadkom klatek podczas animacji.
    \item \textbf{Animacje wejścia}: Zastosowano animację \texttt{@keyframes jumpIn} dla nagłówków, która łączy zmianę przezroczystości z przesunięciem w osi Y, tworząc efekt dynamicznego pojawiania się treści.
    \item \textbf{Interaktywne elementy}: Przyciski, logo oraz sekcje z obrazkami posiadają zdefiniowane stany \texttt{:hover} z płynnymi przejściami (\texttt{transition}), np. efekt skali (\texttt{scale}) czy zmiana filtru szarości (\texttt{grayscale}) na kolor.
\end{itemize}

\section{Mechanizm Scroll Snap (Podstrony)}

Dla podstron zaimplementowano efekt przyciągania przewijania, który dzieli treść na pełnoekranowe sekcje.
\begin{itemize}
    \item Wykorzystano właściwości \texttt{scroll-snap-type: y mandatory} dla kontenera oraz \texttt{scroll-snap-align: start} dla sekcji. Dzięki temu użytkownik podczas przewijania automatycznie "dokuje" do kolejnego logicznego bloku treści, co buduje narracyjny charakter strony.
    \item Każda sekcja posiada wysokość \texttt{100vh} (viewport height), gwarantując wypełnienie całego ekranu.
\end{itemize}

\section{Responsywność (RWD)}

Strona jest w pełni responsywna i dostosowuje się do urządzeń mobilnych oraz tabletów. Wykorzystano podejście priorytetyzujące urządzenia desktopowe z punktami przerwania dla szerokości:
\begin{itemize}
    \item \textbf{1440px}
    \item \textbf{1200px}
    \item \textbf{1024px/900px}
    \item \textbf{600px}
\end{itemize}

Dodatkowe zmiany w układzie mobilnym:
\begin{itemize}
    \item \textbf{Zmiana układu Flexbox:} Wiersze (\texttt{.feature-row}, \texttt{.child-container}) zmieniają kierunek z poziomego na pionowy (\texttt{flex-direction: column}), a elementy układane są jeden pod drugim.
    \item \textbf{Dezaktywacja Scroll Snap}: Na urządzeniach dotykowych o mniejszej rozdzielczości (poniżej 1024px) mechanizm \texttt{scroll-snap} jest wyłączany (\texttt{scroll-snap-type: none}), a wysokość sekcji zmienia się z \texttt{100vh} na \texttt{auto}. Zapobiega to blokowaniu się przewijania na telefonach i poprawia UI/UX.
    \item \textbf{Dostosowanie typografii}: Wielkości czcionek (np. nagłówków Hero) są redukowane, aby mieściły się na węższych ekranach bez przełamywania słów.
    \item \textbf{Nawigacja}: Pasek nawigacji na telefonach zmienia układ na kolumnowy lub "wrap", aby pomieścić wszystkie linki w obszarze dotykowym.
\end{itemize}