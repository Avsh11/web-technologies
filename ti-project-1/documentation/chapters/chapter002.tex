\chapter{Opis założeń projektu}

Celem projektu jest stworzenie responsywnej strony internetowej przy użyciu technologii \texttt{HTML} i \texttt{CSS} dla fikcyjnej korporacji/firmy badawczej zajmującej się nauką, wojskowością oraz technologią w prywatnym sektorze. Strona ma za zadanie zbudować wizerunek nowoczesnej, innowacyjnej marki oraz informować o podstawowych informacjach takich jak, chociażby misji, badaniach i zespole firmy.

Podstawowym problemem, który zostanie rozwiązany przez realizację tego projektu, jest brak cyfrowej tożsamości dla nowo powstałej firmy. Źródłem tego problemu jest współczesna specyfika rynku w dobie powszechnej cyfryzacji, gdzie brak wizytówki w sieci w postaci strony internetowej jest równoznaczny z brakiem wiarygodności w oczach potencjalnych klientów.
Aby problem został skutecznie rozwiązany, potencjalny zespół musi posiadać wiedzę z zakresu technologii webowych w szczególności kaskadowych arkuszy stylów \texttt{CSS3}, języka struktury \texttt{HTML5} oraz podstawowego zmysłu estetycznego w projektowaniu interfejsów UI/UX.

Rozwiązanie problemu przebiegnie w kilku zdefiniowanych krokach. W pierwszej kolejności nastąpi zaprojektowanie schematu każdej ze stron czyt. układ treści i nawigacji. Kolejnym krokiem będzie postawienie szkieletu strony przy pomocy technologii \texttt{HTML5} zaś krokiem kolejnym będzie stopniowe wdrażanie stylów przy użyciu technologii kaskadowych arkuszy stylów. Końcowym i zarazem najważniejszym krokiem będzie optymalizacja responsywności strony na urządzenia mobilne takie jak smartfon bądź tablet poprzez Media Queries.

\section{Wymagania funkcjonalne}
Poniższe wymagania opisują, jakie operacje i interakcje umożliwia strona internetowa użytkownikowi końcowemu:

\textbf{Nawigacja po sekcjach:} Strona musi umożliwiać użytkownikowi płynne przechodzenie po konkretnych sekcjach strony (Home, Our Story, Services, Jobs, Contact) poprzez kliknięcie w odpowiedni załącznik prowadzących do wspomnianych sekcji.

\textbf{Prezentacja treści:} Strona musi wyświetlać sformatowane teksty oraz grafiki w sposób czytelny dla odbiorcy.

\textbf{Interakcja z elementami graficznymi:} Po najechaniu kursorem na zdjęcia bądź elementy paska nawigacji, strona ma odpowiednio reagować poprzez efekty takie jak powiększanie obrazków, podkreślanie elementów i zmiany kolorów.

\textbf{Animacja powitalna:} Podczas załadowania strony głównej, strona automatycznie ma odtworzyć prostą animację powitalną, aby przykuć uwagę użytkownika.

\textbf{Wizualizacja formularza kontaktowego:} Strona musi udostępniać w dolnej części strony głównej formularz kontaktowy z podstawowymi polami na dane z odpowiednio stylowanym interfejsem użytkownika.

\textbf{Responsywność menu:} Na urządzeniach mobilnych strona musi zmieniać układ paska nawigacyjnego na pionowy lub dostosowany do ekranów dotykowych, aby zapewnić podstawową czytelność załączników.\newline\newline

\section{Wymagania niefunkcjonalne}
Poniższe wymagania określają ograniczenia techniczne, standardy oraz cechy, jakimi musi charakteryzować się rozwiązanie:\newline

\textbf{Środowisko testowe}\newline
Testy projektu odbywały się lokalnie na komputerze o przedstawionej poniżej specyfikacji.
\begin{itemize}
      \item \textbf{System operacyjny:} Microsoft Windows 11 Pro
      \item \textbf{Model:} Lenovo Yoga 7 2-in-1 14IML9
      \item \textbf{Procesor:} Intel® Core™ Ultra 5-125H 4.5 GHz
      \item \textbf{Pamięć RAM:} 16GB DDR4
      \item \textbf{Karta graficzna:} Intel® Arc™ graphics (Zintegrowana)
      \item \textbf{Dysk:} Samsung MZAL81T0HDLB-00BL2 1TB
      \item \textbf{Rozdzielczość ekranu:} 2880 x 1800
      \item \textbf{IDE:} Visual Studio Code
\end{itemize}

\textbf{Technologie:} Strona zgodnie z przyjętymi wytycznymi wykonana ma być przy użyciu czystego \texttt{HTML5} oraz \texttt{CSS} bez zwenętrznych bibliotek, frameworków (np. Bootstrap, React itd) oraz języka \texttt{JavaScript}.

\textbf{Ogólna responsywność:} Strona musi poprawnie skalować się i być czytelna na urządzeniach zarówno mobilnych aż po monitory desktopowe.

\textbf{Estetyka i spójność wizualna:} Strona musi utrzymywać spójność kolorystyczną, wykorzystując spójną typografię w zależności od sekcji.

\textbf{Wydajność:} Strona powinna charakteryzować się krótkim czasem ładowania dzięki braku ciężkich bibliotek i zewnętrznych skryptów.

\textbf{Kompatybilność:} Witryna powinna wyświetlać się poprawnie zarówno na przeglądarkach wyposażonych w silnik \texttt{Chromium} jak i \texttt{Gecko}.

\textbf{Czytelność kodu:} Kod źródłowy (HTML i CSS) musi być uporządkowany, posiadać wcięcia oraz komentarze ułatwiające późniejszą ocenę, jak i edycje.

