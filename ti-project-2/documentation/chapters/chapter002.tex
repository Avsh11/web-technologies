\chapter{Opis założeń projektu}

Celem projektu było stworzenie wielofunkcyjnej strony internetowej o nazwie \textbf{Vexillarius}, będącej praktycznym sprawdzianem podstawowych umiejętności głównie z zakresu technologii \textbf{JavaScript}. Strona została zaprojektowana w sposób spójny wizualnie z myślą o zaimplementowanych skryptach.

Głównym założeniem projektu było przejście ze statycznego dokumentu HTML/CSS do interaktywnej witryny poprzez użycie skryptów. W ramach prac zaimplementowano ich kilka, przy czym część z nich to narzędzia takie jak kalkulator oraz lista zadań, które pozwalają na interakcję z użytkownikiem.

Rozwiązanie problemu projektowego przebiegło w następujących krokach:
\begin{enumerate}
      \item Przygotowanie struktury dokumentu \texttt{HTML5}.
      \item Stworzenie stylów \texttt{CSS3} wraz z tymi odpowiedzialnymi za podstawową responsywność witryny - \texttt{Media Queries}.
      \item Implementacja skryptów w języku \texttt{JavaScript} odpowiedzialnych za warstwę użytkową strony.
\end{enumerate}

\section{Wymagania funkcjonalne}
Projekt realizuje następujące funkcjonalności, które pozwalają użytkownikowi na interakcję z witryną:

\textbf{Przełącznik trybu light/dark mode:} Możliwość zmiany motywu strony (jasny/ciemny) za pomocą jednego przycisku zlokalizowanego w prawym górnym rogu paska nawigacyjnego.

\textbf{Interaktywna lista zadań (To-Do):} Lista pozwala użytkownikowi na wpisywanie i dodawanie zadań, jak i ich usuwanie za pomocą kliknięcia. Lista zadań wyświetlana jest za pomocą dynamicznej listy na ekranie.

\textbf{Kalkulator liczb - arabskie/rzymskie:} Kalkulator przeliczający wpisane liczby arabskie na ich rzymskie odpowiedniki, prezentujący wynik w czasie rzeczywistym.

\textbf{Obsługa formularza kontaktowego:} Sekcja z walidacją pól (imię, e-mail, treść wiadomości). Skrypt sprawdza poprawność danych i informuje użytkownika o błędach lub powodzeniu wysłania wiadomości za pomocą autorskich komunikatów - \textit{custom alerts}.

\textbf{Dynamiczny pasek nawigacji:} Nagłówek strony podczas scrollowania w dół zmienia swój wygląd poprzez pojawienie się dolnego obramowania oddzielającego go od reszty strony.

\section{Wymagania niefunkcjonalne}

\textbf{Środowisko testowe}\newline
Testy projektu odbywały się lokalnie na komputerze o przedstawionej poniżej specyfikacji.
\begin{itemize}
      \item \textbf{System operacyjny:} Microsoft Windows 11 Pro
      \item \textbf{Model:} Lenovo Yoga 7 2-in-1 14IML9
      \item \textbf{Procesor:} Intel® Core™ Ultra 5-125H 4.5 GHz
      \item \textbf{Pamięć RAM:} 16GB DDR4
      \item \textbf{Karta graficzna:} Intel® Arc™ graphics (Zintegrowana)
      \item \textbf{Dysk:} Samsung MZAL81T0HDLB-00BL2 1TB
      \item \textbf{Rozdzielczość ekranu:} 2880 x 1800
      \item \textbf{IDE:} Visual Studio Code
\end{itemize}

\textbf{Technologie:} Strona zgodnie z przyjętymi wytycznymi powinna skupiać się na wykorzystaniu skryptów w języku \textbf{JavaScript}. Dozwolone jest natomiast wykorzystanie ich w formie strony internetowej, gdzie wykorzystane będą również inne technologie.

\textbf{Wydajność:} Strona powinna charakteryzować się krótkim czasem ładowania poprzez brak ciężkich bibliotek.

\textbf{Kompatybilność:} Witryna powinna wyświetlać się poprawnie zarówno na przeglądarkach wyposażonych w silnik \texttt{Chromium}, jak i \texttt{Gecko}.

\textbf{Responsywność (RWD):} Strona poprawnie wyświetla się na różnych szerokościach ekranu - od monitorów aż po urządzenia mobilne. Układ elementów dostosowuje się tak, aby zachować czytelność i wygodę klikania.

\textbf{Spójność wizualna:} Wykorzystanie zmiennych globalnych w kaskadowych arkuszach stylów w celu zarządzania kolorystyką oraz typografią nadaje stronie estetycznego, jednolitego wyglądu.
