\chapter{Harmonogram realizacji projektu}
\section{Etapy realizacji projektu}
Proces tworzenia witryny był ciągiem mniejszych bądź większych etapów, takich jak:
\begin{enumerate}
    \item Analiza wymagań i ułożenie wstępnego schematu.
    \item Wstępne stworzenie układu strony.
    \item Implementacja rozwiązania przy użyciu technologii \texttt{HTML5}.
    \item Wprowadzenie kaskadowych arkuszy stylów.
    \item Dodanie skryptów \texttt{JavaScript}.
    \item Drobne zmiany wizualne, takie jak dodanie drobnego skryptu dla paska nawigacyjnego.
    \item Stworzenie dokumentacji technicznej.
\end{enumerate}
\section{System kontroli wersji}
\begin{itemize}
    \item Podczas tworzenia projektu wykorzystano system kontroli wersji \textbf{Git} poprzez aplikację \textbf{GitHub Desktop}. Do przechowywania kodu źródłowego wykorzystano repozytorium na platformie \textbf{GitHub}. Repozytorium dostępne jest pod adresem:
    \begin{center}
    \textbf{\texttt{\url{https://github.com/Avsh11/web-technologies/tree/main/ti-project-2}}}
    \end{center}
    \item Repozytorium zawiera pełną wersję projektu po ostatecznych poprawkach (w chwili oddania) i dostępne będzie w trybie publicznym ustalonym przez wykładowcę prowadzącego przedmiot.
\end{itemize}