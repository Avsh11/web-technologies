\chapter{Opis struktury projektu}

\section{Użyte technologie}
Strona internetowa \textbf{Vexillarius} została wykonana przy użyciu natywnych technologii webowych. Zgodnie z założeniami projektowymi, zrezygnowano z zewnętrznych frameworków na rzecz czystego kodu.

\texttt{HTML5} wykorzystano do stworzenia podstawowego szkieletu witryny. Zastosowanie znaczników takich jak \texttt{<header>}, \texttt{<nav>}, \texttt{<section>} oraz \texttt{<footer>} zapewnia poprawną strukturę dokumentu i wspiera jego dostępność (accessibility).

\texttt{CSS} wykorzystano do stworzenia warstwy wizualnej opartej na nowoczesnych standardach, w tym Flexbox do układów oraz zmienne do prostego zarządzania kolorami i typografią.

\texttt{JavaScript} wykorzystano do stworzenia skryptów odpowiadających za funkcjonalność całej strony. Zastosowano DOM (Document Object Model) oraz obsługę zdarzeń w czasie rzeczywistym.
\section{Struktura katalogów}
Projekt został podzielony na katalogi, co zapewnia separację sturktury, stylów oraz skryptów. Ułatwia to konserwację kodu oraz jego ewentualną rozbudowę. 
\vspace{0.5cm} 
\dirtree{%
.1 ti-project-2/.
.2 css/ \DTcomment{Arkusze stylów}.
.3 style.css \DTcomment{Główny arkusz stylów witryny}.
.2 documentation/ \DTcomment{Pliki źródłowe dokumentacji}.
.2 html/ \DTcomment{Struktura dokumentów}.
.3 index.html \DTcomment{Główny plik strony internetowej}.
.2 javascript/ \DTcomment{Logika aplikacji (skrypty)}.
.3 animation.js \DTcomment{Obsługa animacji i efektów scrolla}.
.3 calc.js \DTcomment{Logika kalkulatora rzymskiego}.
.3 form.js \DTcomment{Obsługa formularza i walidacji}.
.3 theme.js \DTcomment{Skrypt zmiany motywu kolorystycznego}.
.3 todo.js \DTcomment{Obsługa interaktywnej listy zadań}.
}
\vspace{0.5cm}
\section{Konwencje kodowania i organizacja stylów}
W arkuszach stylów zastosowano konwencje zwiększające czytelność i uniwersalność kodu:
\begin{itemize}
    \item \textbf{Zmienne CSS (:root)}: Definicje kolorów (tła, tekstu, akcentów) zostały wraz z typografią zamknięte w zmiennych. Pozwala to na błyskawiczną zmianę globalnej palety barw oraz umożliwiło Implementację trybu \texttt{dark-mode} poprzez nadpisywanie wartości zmiennych w klasie nadrzędnej.
    \item \textbf{Skalowalne jednostki REM}: Bazowa wielkość czcionki została ustawiona na \texttt{50\%} (8px). Dzięki temu projekt jest łatwiej skalowalny, a jednostki \texttt{REM} pozwalają na zachowanie proporcji między elementami.
    \item \textbf{Box-siging}: Globalne ustawienie \texttt{border-box} zapobiega problemom z obliczaniem szerokości elementów przy dodawaniu marginesów wewnętrznych (padding).
\end{itemize}
\section{Implementacja logiki w JavaScript}
Skrypty projektu zostały zaprojektowane w sposób modułowy. Każda funkcjonalność znajduje się w osobnym pliku, co zapobiega konfliktom nazw i ułatwia ewentualne debugowanie:
\begin{itemize}
\item \textbf{Manipulacja DOM}: Skrypty dynamicznie tworzą (np. lista zadań) lub modyfikują (np. alerty) elementy drzewa dokumentu.
\item \textbf{Event Listeners}: Interakcja z użytkownikiem opiera się na nasłuchiwaniu zdarzeń takich jak \texttt{click}, \texttt{scroll} czy \texttt{keypress}.
\item \textbf{Powiadomienia}: Zastosowano funkcje czasu (\texttt{setTimeout}), aby kontrolować czas wyświetlania komunikatów błędów i sukcesów w przypadku formularza kontaktowego.
\end{itemize}
\section{Responsywność (RWD)}
Strona dostosowuje się do urządxeń mobilnych poprzez \textit{Media Queries}. Zastosowano podejście które zakłada płynne przejścia między szerokościami tak aby strona wyglądała czytelnie na urządzeniach węższych.
\begin{itemize}
    \item \textbf{Typografia:} Wykorzystano funkcję \texttt{clamp()}, która automatycznie dobiera wielkość czcionki w zależności od szerokości okna przeglądarki.
    \item \textbf{Pasek nawigacji:} Pasek nawigacji na telefonach zmienia swój układ na kolumnowy.
    \item \textbf{Flex-column:} Na mniejszych ekranach kontenery formularza i narzędzi (kalkulator, lista) zmieniają orientacje z poziomej na pionową.
\end{itemize}
\section{Prezentacja najważniejszych skryptów i ich fragmentów}
W niniejszej sekcji przedstawiono implementację oraz krótką analizę naważniejszych skryptów \textbf{JavaScript}. Każdy fragment kodu został opatrzony krótkim komentarzem technicznym wyjaśniającym zasadę jego działania.
\subsection{Zarządzanie motywem (theme.js)}
Skrypt ten odpowiada za interakcję z użytkownikiem w zakresie wyboru preferowanego motywu wizualnego (Dark Mode / Light Mode).
\begin{center}
\begin{minipage}{0.85\textwidth}
\begin{lstlisting}[style=styleEditorJS, caption={Przełącznik motywów Dark Mode / Light Mode}]
var themeButton = document.getElementById('theme-button');
themeButton.onclick = function() {
var pageBody = document.body;
pageBody.classList.toggle('dark-mode');
};
\end{lstlisting}
\end{minipage}
\end{center}
\textbf{Wyjaśnienie:} Funkcja przypisana do zdarzenia \texttt{onclick} przycisku wywołuje metodę \texttt{toggle()} na liście klas dokumentu. Mechanizm automatycznie dodaje klasę \texttt{.dark-mode} przy jej braku lub usuwa ją, gdy jest już ona obecna. Powoduje to natycmiastową aktualizację wartości zmiennych CSS.
\subsection{Animacja powitalna sekcji Hero (animation.js)}
Skrypt odpowiedzialny za pojawienie się tytułu strony wykorzystuje funkcję czasu do płynnej zmiany właściwości CSS.
\begin{center}
\begin{minipage}{0.85\textwidth}
\begin{lstlisting}[style=styleEditorJS, caption={Implementacja animacji "fade-in-up"}]
var title = document.querySelector('.hero-content h1');
var position = 50;
var opacity = 0;
title.style.marginTop = position + "px";
title.style.opacity = opacity;
var timer = setInterval(function() {
position = position - 1;
opacity = opacity + 0.02;
title.style.marginTop = position + "px";
title.style.opacity = opacity;

if (position <= 0) {
    clearInterval(timer);
    title.style.marginTop = "0px";
    title.style.opacity = "1";
}
}, 15);
\end{lstlisting}
\end{minipage}
\end{center}
\textbf{Wyjaśnienie:} Skrypt ręcznie steruje animacją przy użyciu metody \texttt{setInterval}, która wykonuje blok co 15 milisekund. W każdym kroku wartość zmiennej \texttt{position} maleje powodując ruch tekstu w górę, zaś \texttt{opacity} rośnie zwiększając widoczność. Po osiągnięciu docelowej pozycji, interwał jest czyszczony przez \texttt{clearInterval}, co zatrzymuje proces.
\subsection{Pasek nawigacji (animation.js)}
Skrypt ten zapewnia wizualne oddzielenie paska nawigacji od treści strony podczas jej przewijania.
\begin{center}
\begin{minipage}{0.85\textwidth}
\begin{lstlisting}[style=styleEditorJS, caption={Zmiana stylu nagłówka przy przewijaniu}]
var header = document.querySelector('header');
window.addEventListener('scroll', () => {
if (window.scrollY > 0) {
header.classList.add('scrolled');
} else {
header.classList.remove('scrolled');
}
});
\end{lstlisting}
\end{minipage}
\end{center}
\textbf{Wyjaśnienie:} Skrypt rejestruje słuchacza zdarzeń (\texttt{Event Listener}) dla obiektu \texttt{window}. Gdy wartość \texttt{scrollY} przekracza 0, do nagłówka dodawana jest klasa \texttt{.scrolled}, która w CSS odpowiada za wyświetlenie dolnego obramowania \texttt{border-bottom}.
\subsection{Konwerter liczb rzymskich (calc.js)}
Skrypt implementuje algorytm przeliczania liczb arabskich na liczby systemu rzymskiego. Z oczywistych względów nie wylistowano obsługi przycisków i deklaracji zmiennych. 
\begin{center}
\begin{minipage}{0.85\textwidth}
\begin{lstlisting}[style=styleEditorJS, caption={Logika przeliczania liczb arabskich na rzymskie}]
function convert() {
var val = parseInt(numberInput.value);
if (isNaN(val) || val <= 0 || val > 3999) {
    out.innerText = "Error";
    return;
}

var numbers = [1000, 900, 500, 400, 100, 90, 50, 40, 10, 9, 5, 4, 1];
var roman = ["M", "CM", "D", "CD", "C", "XC", "L", "XL", "X", "IX", "V", "IV", "I"];
var finalStr = "";

for (var i = 0; i < numbers.length; i++) {
    while (val >= numbers[i]) {
        finalStr = finalStr + roman[i];
        val = val - numbers[i];
    }
}
out.innerText = finalStr;
}
\end{lstlisting}
\end{minipage}
\end{center}
\textbf{Wyjaśnienie:} Algorytm sprawdza poprawność danych wejściowych (Zakres 1 - 3999), a następnie wykonuje pętle \texttt{for} do iteracji po tablicy zdefiniowanych progów liczbowych. Wewnętrzna pętla \texttt{while} odejmuje najwyższą możliwą wartość od liczby wejściowej i dopisuje odpowiadający jej znak rzymski do wyniku.
\subsection{Interaktywna lista zadań (todo.js)}
Skrypt demonstruje działanie listy zadań.
\begin{center}
\begin{minipage}{0.85\textwidth}
\begin{lstlisting}[style=styleEditorJS, caption={Dynamiczne dodawanie i usuwanie zadań}]
function addNewTask() {
var text = inputField.value;
if (text != "") {
    var newItem = document.createElement('li');
    newItem.className = 'todo-item';
    newItem.innerHTML = "<span>" + text + "</span>";

    newItem.onclick = function() {
        this.remove();
    };

    myList.appendChild(newItem);
    inputField.value = ""; 
}
}
\end{lstlisting}
\end{minipage}
\end{center}
\textbf{Wyjaśnienie:} Po aktywacji funkcji, skrypt tworzy w pamięci nowy element \texttt{li}, nadaje mu klasę stylu i wstrzykuje tekst pobrany z inputa. Każdemu elementowi przypisywana jest również funkcja \texttt{remove()}, wyzwalana po kliknięciu w dodany wcześniej element, co umożliwia selektywne usuwanie z listy.
\subsection{Obsługa formularza i system alertów (form.js)}
Skrypt odpowiada za weryfikację danych w formularzu oraz za wysyłanie powiadomień w przypadku błędnych danych lub poprawnie przesłanego formularza.
\begin{center}
\begin{minipage}{0.85\textwidth}
\begin{lstlisting}[style=styleEditorJS, caption={Walidacja w formularzu i generowanie powiadomień}]
function myAlert(msg, mode) {
    var old = document.querySelector('.custom-alert');
    if (old) { old.remove(); }

    var box = document.createElement('div');
    box.className = 'custom-alert ' + mode;
    box.innerHTML = "<span>" + (mode === 'success' ? 'EMOTKA1' : 'EMOTKA2') + "</span> " + msg;
    document.body.appendChild(box);

    setTimeout(function() { box.classList.add('show'); }, 50);
    setTimeout(function() { box.classList.remove('show'); }, 3000);
}

var contactForm = document.getElementById('contact-form');

if (contactForm) {
    contactForm.onsubmit = function(e) {
        e.preventDefault();
        var userEmail = document.getElementById('email').value;
        
        if (userEmail.indexOf("@") == -1 || userEmail.length < 5) {
            myAlert("Adres email jest nieprawidłowy!", "error");
            return;
        }
        myAlert("Wiadomość została przesłana.", "success"); 
        contactForm.reset();
    };
}
\end{lstlisting}
\end{minipage}
\end{center}
\textbf{Wyjaśnienie:} Funkcja \texttt{myAlert} dynamicznie buduje okno powiadomienia w drzewie DOM, wykorzystując funkcję \texttt{setTimeout} do sterowania animacją pojawiania się i znikania klasy styulu. Skrypt do walidacji danych formularza blokuje domyślną wysyłkę formularza przy użyciu \texttt{e.preventDefault()} i przeprowadza test poprawności adres e-mail, imienia, wiadomości wywołując alert z odpowiednim statusem sukcesu albo błędu.