\chapter{Harmonogram realizacji projektu}
\section{Etapy realizacji projektu}
Proces tworzenia witryny był ciągiem mniejszych bądź większych etapów takich jak:
\begin{enumerate}
    \item Analiza wymagań i ułożenie wstępnego schematu.
    \item Wstępne stworzenie układu strony.
    \item Implementacja rozwiązania przy użyciu technologii \texttt{HTML5}
    \item Wprowadzenie kaskadowych arkuszy stylów.
    \item Przeprowadzenie ponownego ułożenia witryny na nowo - reset.
    \item Ukończenie strony głównej.
    \item Ponowienie kroków do pozostałych podstron zaczynając od ułożenia schematu oraz układu kończąc na implementacji \texttt{CSS} w celu dodania animacji, wejścia układu strony w życie i manipulacji treścią.
    \item Zakończenie prac nad podstronami.
    \item Przerobienie strony głównej.
    \item Drobne poprawki.
    \item Stworzenie dokumentacji technicznej.
\end{enumerate}
\section{Wykres Gantta}
% Wykres Gantta przedstawiony jest na rysunku poniżej - \textbf{rys. \ref{rys 4.1}}

% \begin{figure}[H]
%     \centering
%     \includegraphics[width=\linewidth]{figures/}\\
%     \caption{Wykres Gantta przedstawiający harmonogram realizacji projektu}\label{rys 4.1}
% \end{figure}

\section{System kontroli wersji}
\begin{itemize}
    \item Podczas tworzenia projektu wykorzystano system kontroli \textbf{Git} poprzez aplikacje \textbf{GitHub Desktop}. Do przechowywania kodu źródłowego wykorzystano repozytorium na platformie \textbf{GitHub}. Repozytorium dostępne jest pod adresem:
    \begin{center}
    \textbf{\texttt{\url{https://github.com/Avsh11/web-technologies/tree/main/ti-project-1}}}
    \end{center}
    \item Repozytorium zawiera pełną wersje projektu po ostatecznych poprawkach (w chwili oddania) i dostępne będzie w trybie publicznym ustalonym przez wykładowcę prowadzącego przedmiot.
\end{itemize}